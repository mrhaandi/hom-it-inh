\documentclass[a4paper,11pt]{article}
\usepackage[T1]{fontenc}\usepackage[utf8]{inputenc}
%%%%%%%%%%%%%%%%%%%%%%%%% Packages %%%%%%%%%%%%%%%%%%%%%%%%%%%%%%%%%%%%%%%%%
\usepackage{amssymb}
\usepackage{amsmath}
\usepackage{amsthm}
\usepackage{a4wide,xspace}
\usepackage{enumitem}
\usepackage{amsmath}
\usepackage{csquotes} %\enquote{} macro
\usepackage{lmodern} %easier on the eyes modern font
\usepackage{hyperref} %linked references
\usepackage{lineno}\linenumbers
\usepackage{wasysym} %\brokenvert
\usepackage{bussproofs}
\usepackage{listings}
%\usepackage[activate={true,nocompatibility}]{microtype} %layouting
%%%%%%%%%%%%%%%%%%%%%%% Style changes %%%%%%%%%%%%%%%%%%%%%%%%%%%%%%%%%%%
\theoremstyle{definition}
\newtheorem{theorem}{Theorem}
\newtheorem{lemma}[theorem]{Lemma}
\newtheorem{fact}[theorem]{Fact}
\newtheorem{corollary}[theorem]{Corollary}
\newtheorem{example}[theorem]{Example}
\newtheorem{claim}[theorem]{Claim}
\newtheorem{proposition}[theorem]{Proposition}
\newtheorem{remark}[theorem]{Remark}
\newtheorem{conjecture}[theorem]{Conjecture}
\newtheorem{definition}[theorem]{Definition}
\newtheorem{problem}[theorem]{Problem}
%%%%%%%%%%%%%%%%%%%%%%%%%%%% Symbols %%%%%%%%%%%%%%%%%%%%%%%%%%%%%%%%%%%

%%%%%%%%%%%%%%%%%% specific %%%%%%%%%%%%%%%%%%%%%%%%%%%%%%%%%%%%%%%%%%%%%%%
\DeclareMathOperator{\id}{id}
\DeclareMathOperator{\length}{length}
\DeclareMathOperator{\depth}{depth}


%%%%%%%%%%%%%%%%%%%%%%%% stylized symbols %%%%<%%%%%%%%%%%%%%%%%%%%%%%%%%%%%%%%%%%%
\newcommand{\bbB}{\mathbb{B}}
\newcommand{\bbC}{\mathbb{C}}
\newcommand{\bbN}{\mathbb{N}}
\newcommand{\bbT}{\mathbb{T}}
\newcommand{\bbS}{\mathbb{S}}
\newcommand{\bbV}{\mathbb{V}}
\newcommand{\bbZ}{\mathbb{Z}}
\newcommand{\calC}{\mathcal{C}}
\newcommand{\calD}{\mathcal{D}}
\newcommand{\calF}{\mathcal{F}}
\newcommand{\calM}{\mathcal{M}}
\newcommand{\calR}{\mathcal{R}}
\newcommand{\calT}{\mathcal{T}}
\newcommand{\calU}{\mathcal{U}}

\newcommand{\0}{\mathbf{0}}
\newcommand{\1}{\mathbf{1}}

\newcommand{\case}[2]{\ensuremath{\textbf{case } #1 \textbf{ of } \langle #2 \rangle}}
\newcommand{\caseelse}[3]{\ensuremath{\textbf{case } #1 \textbf{ of } \langle #2 \rangle \textbf{ else } #3}}
\newcommand{\ite}[4]{\ensuremath{\textbf{if } #1 \textbf{ is } #2 \textbf{ then } #3 \textbf{ else } #4}}
\newcommand{\itb}[3]{\ensuremath{\textbf{if } #1 \textbf{ is } #2 \textbf{ then } #3}}
%%%%%%%%%%%%%%%%%%%%%%%%% Abbreviations etc. %%%%%%%%%%%%%%%%%%%%%%%%%%%%%%%%

\newcommand{\ga}{\ensuremath{\iota}}

%%%%%%%%%%%%%%%%%%%%%%%%%%%%%%%%%%%%%%%%%%%%%%%%%%%%%%%%%%%%%%%%%%%%%%%%%%%%%
\begin{document}
\title{Higher-order Matching -- a Loader and Urzyczyn Story}
\author{Andrej Dudenhefner\\ TU Dortmund University, Germany \\
\texttt{andrej.dudenhefner@cs.tu-dortmund.de}}
\date{\today}
\maketitle

\begin{abstract}
Thoughts on undecidability of higher-order matching wrt.~$(=_\beta)$ (not to be confused with~$(=_{\beta\eta})$) using intersection type inhabitation.
\end{abstract}


\section{Preliminaries}

\subsection*{Simply-typed $\lambda$-Calculus}

\begin{definition}[$\lambda$-Terms]
\[
\begin{array}{rcl}
M, N &::= &x \mid M \, N \mid \lambda x.M
\end{array}
\]
\end{definition}

\begin{definition}[$\beta$-Reduction]~
\begin{itemize}
\item $\to_\beta$ is the contextual closure of $(\lambda x.M)\,N \to_\beta M[x := N]$
\item $\twoheadrightarrow_\beta$ is the reflexive, transitive closure of $\to_\beta$
\item $=_\beta$ is the reflexive, transitive, symmetric closure of $\to_\beta$
\end{itemize}
\end{definition}

\begin{definition}[Simple Types with Ground Atom $\ga$]
\[
\begin{array}{rcl}
\sigma, \tau &::= &\ga \mid \sigma \to \tau
\end{array}
\]
\end{definition}

\begin{definition}[Type Environments]
\[
\begin{array}{rcl}
\Gamma &::= &\{x_1 : \sigma_1, \ldots, x_n : \sigma_n\}
\end{array}
\]
\end{definition}

\begin{definition}[Simple Type System]~\\
\begin{tabular}{ccc}
{%\RightLabel{\textnormal{(Var)}}
\AxiomC{$(x : \sigma) \in \Gamma$}
\UnaryInfC{$\Gamma \vdash x : \sigma$}
\DisplayProof}
&
{%\RightLabel{\textnormal{($\to$I)}}
\AxiomC{$\Gamma, x : \sigma \vdash M : \tau$}
\UnaryInfC{$\Gamma \vdash \lambda x.M : \sigma \to \tau$}
\DisplayProof}
&
{%\RightLabel{\textnormal{($\to$E)}}
\AxiomC{$\Gamma \vdash M : \sigma \to \tau$}
\AxiomC{$\Gamma \vdash N : \sigma$}
\BinaryInfC{$\Gamma \vdash M\,N : \tau$}
\DisplayProof}
\end{tabular}
\end{definition}

\subsection*{Higher-order $\beta$-Matching}

\begin{problem}[Higher-order $\beta$-Matching (${F}\,\mathsf{X} = N$)]
Given terms $F, N$ and simple types $\sigma, \tau$ such that $\emptyset \vdash F : \sigma \to \tau$ and $\emptyset \vdash N : \tau$, is there a term $M$ such that $\emptyset \vdash M : \sigma$ and $F\,M =_\beta N$?
\end{problem}

\begin{theorem}[{\cite[Theorem~5.5]{Loader03}}]
Higher-order $\beta$-matching is undecidable.
\end{theorem}

\begin{proof}[Proof Outline]
Many-one reduction from an ad-hoc variant of $\lambda$-definability.
\end{proof}

\begin{remark}
Artifacts
\begin{itemize}
\item \emph{myopic} $\lambda\bot$-terms and \emph{myopic} order
\item characterization of $1$-lifts and $7$-lifts by \emph{checker} $\lambda\bot$-terms
\item characterization of finite models by $(7+1)$-lifted terms\qedhere
\end{itemize}
\end{remark}

\subsection*{Finite Models}

\begin{definition}[Finite Model]
Fix a finite set $\calM_\ga$, $\calM_{\sigma \to \tau}$ is the finite set of all functions from $\calM_\sigma$ to $\calM_\tau$.
\end{definition}

We use tabular notation for families of finite functions in a given finite model, illustrated by the following examples.

\setlength{\arraycolsep}{1pt}

\begin{example}
Fix the finite model $\calM$ with $\calM_\ga = \{1, 2\}$.
The family ${\small\left(\begin{array}{c}
1 \mapsto 2\\
2 \mapsto 1
\end{array}\right)} \subseteq \calM_{\ga \to \ga}$ contains exactly one finite function $f \in \calM_{\ga \to \ga}$ where $f(1) = 2$, $f(2) = 1$.
\end{example}

\begin{example}
Fix the finite model $\calM$ with $\calM_\ga = \{1, 2, 3\}$.
The family ${\small\left(\begin{array}{c}
1 \mapsto 2\\
2 \mapsto 3
\end{array}\right)} \subseteq \calM_{\ga \to \ga}$ contains exactly three distinct finite functions $f \in \calM_{\ga \to \ga}$ where $f(1) = 2$, $f(2) = 3$, and $f(3) \in \{1, 2, 3\}$.
\end{example}

\begin{example}\label{xmp:fin-model}
Fix the finite model $\calM$ with $\calM_\ga = \{1, 2, 3\}$. The family $\left({\small\left(\begin{array}{c}
1 \mapsto 2\\
2 \mapsto 3
\end{array}\right)} \mapsto (1 \mapsto 3)\right)$ contains finite functions $f \in \calM_{(\ga \to \ga) \to \ga \to \ga}$ such that $f(g)(1) = 3$ for any $g \in {\small\left(\begin{array}{c}
1 \mapsto 2\\
2 \mapsto 3
\end{array}\right)}$.

As a side note, there is a $\lambda$-definable member of $\left({\small\left(\begin{array}{c}
1 \mapsto 2\\
2 \mapsto 3
\end{array}\right)} \mapsto (1 \mapsto 3)\right)$ realized by the $\lambda$-term $\lambda f.\lambda x.f\,(f\,x)$.
\end{example}

\subsection*{Simple Semi-Thue Systems}

\begin{definition}[Simple Semi-Thue System]
A semi-Thue system $\calR$ is \emph{simple}, if each rule has the form $ab \Rightarrow cd$.
\end{definition}

\begin{remark}
Simple Semi-Thue System inspired by \enquote{\emph{Inhabitation of Low-Rank Intersection Types}}~[\cite{Urzyczyn09}]
and \enquote{\emph{Loader and Urzyczyn Are Logically Related}}~[\cite{SalvatiMGB12}].
\end{remark}


\begin{example}
Let $\calR  := \{00 \Rightarrow 21, 02 \Rightarrow 11\}$ over the alphabet $\{0,1,2\}$.\medskip

We have $000 \Rightarrow_\calR 021 \Rightarrow_\calR 111$.
\end{example}

\begin{problem}[$0^+ \Rightarrow^*_\calR 1^+$]\label{prb:ssts01}
Given a simple semi-Thue system $\calR$, does $0^n \Rightarrow^*_\calR 1^n$ hold for some $n > 0$?
\end{problem}

\begin{remark}
Problem~\ref{prb:ssts01} used as a starting point in~\cite{DudenhefnerR19} by Dudenhefner.
\end{remark}

\begin{theorem}[{\cite[Lemma~3.3]{DudenhefnerR19}}]
Problem $0^+ \Rightarrow^*_\calR 1^+$ is undecidable.
\end{theorem}

\begin{remark}~
\begin{itemize}
\item similar to \cite[Lemma~2]{Urzyczyn09}
\item mechanized in Coq~\cite[\texttt{SSTS01}]{CLUP20}
\end{itemize}
\end{remark}

\newpage

\section{Undecidability of Higher-order $\beta$-Matching}

New plan (no intersection types):

Given a simple semi-Thue system $\calR$

\begin{enumerate}
\item construct a higher-order $\beta$-matching instance $F_\calR \, \textsf{X} = N_\calR$
\begin{itemize}
\item $F_\calR \, \textsf{X} = N_\calR$ is solvable iff $0^+ \Rightarrow^*_\calR 1^+$
\end{itemize}
\begin{enumerate}
\item Syntactic constraints
\item Semantic constraints
\end{enumerate}
\item mechanize the construction in Coq ($4000$ LOC)
\end{enumerate}

Two examples.
First, in which there is only a semantic constraint and where the solution is wrong and arbitrary syntax.

\begin{example}
Consider the finite model $\calM$ with $\calM_\ga = \{1, 2, 3\}$ from Example~\ref{xmp:fin-model} together with the family $\calF = \left({\small\left(\begin{array}{c}
1 \mapsto 2\\
2 \mapsto 3
\end{array}\right)} \mapsto (1 \mapsto 3)\right)$ of finite functions $f \in \calM_{(\ga \to \ga) \to \ga \to \ga}$ such that $f(g)(1) = 3$ for any $g \in {\small\left(\begin{array}{c}
1 \mapsto 2\\
2 \mapsto 3
\end{array}\right)}$.

Let us realize elements of the set $\{1, 2, 3\}$ by projections $\pi_1 := \lambda xyz.x$, $\pi_2 := \lambda xyz.y$, and $\pi_3 := \lambda xyz.z$ respectively.
For the term $G := \lambda h.\lambda x_1 x_2 x_3.h\,x_2\,x_3\,x_1$ we have
\[G\,\pi_1 \twoheadrightarrow_\beta \pi_2 \text{ and } G\,\pi_2 \twoheadrightarrow_\beta \pi_3\]
Therefore, $G$ realizes a member of ${\small\left(\begin{array}{c}
1 \mapsto 2\\
2 \mapsto 3
\end{array}\right)}$.
Consider the matching constraint $\mathsf{X}\,G\,\pi_1 =_\beta \pi_3$.
Any term which realizes a member of the family $\calF$ solves the above constraint.
For instance, we have $\big(\lambda f.\lambda x.f\,(f\,x)\big)\,G\,\pi_1 \twoheadrightarrow_\beta \pi_3$.

However, the converse is not the case.
One obvious ad-hoc solution to the above constraint is $\big(\lambda f.\lambda x.\pi_3\big)\,G\,\pi_1
\twoheadrightarrow_\beta \pi_3$.
Another, more \enquote{troubling} solution to he above constraint is $\big(\lambda f.\lambda x.\lambda x_1 x_2 x_3.x\,x_3\,x_3\,x_3\big)\,G\,\pi_1
\twoheadrightarrow_\beta \lambda x_1 x_2 x_3.\pi_1\,x_3\,x_3\,x_3 \twoheadrightarrow_\beta \pi_3$.
The reason for such solutions is an ad-hoc construction conjuring up universe elements. 
\end{example}

Second, we add syntactic constraint so that the solutions are well-behaved.

\begin{example}
TODO
\end{example}

\subsection{Syntactic Constrains}
We present two constraints such that solutions are of certain shape.

\subsection{Semantic Constrains}

\begin{example}~
\begin{itemize}
\item specification $A := \big((1 \to 2) \wedge (2 \to 3) \wedge (3 \to 1)\big) \to 1 \to 3$
\item universe elements
\begin{itemize}
\item $\pi_1 := \lambda xyz.x$
\item $\pi_2 := \lambda xyz.y$
\item $\pi_3 := \lambda xyz.z$
\item $\emptyset \vdash \pi_i : \sigma_\calU$ where $\sigma_\calU := \ga \to \ga \to \ga \to \ga$
\end{itemize}
\item permutation $G := \lambda h.\lambda xyz.h\,y\,z\,x$
\begin{itemize}
\item $G\,\pi_1 \twoheadrightarrow_\beta \pi_2$
\item $G\,\pi_2 \twoheadrightarrow_\beta \pi_3$
\item $G\,\pi_3 \twoheadrightarrow_\beta \pi_1$
\item[$\leadsto$] $G$ realizes specification $(1 \to 2) \wedge (2 \to 3) \wedge (3 \to 1)$
\item $\emptyset \vdash G : \sigma_\calU \to \sigma_\calU$
\end{itemize}
\item matching constraint $X\,G\,\pi_1 =_\beta \pi_3$
\item \emph{well-behaved} solution %$X \mapsto \tm{\lambda g.\lambda h.g\,(g\,h)}$
\begin{itemize}
\item $X\,G\,\pi_1 \mapsto \big(\lambda g.\lambda h.g\,(g\,h)\big)\,G\,\pi_1
\twoheadrightarrow_\beta G\,(G\,\pi_1) \twoheadrightarrow_\beta \pi_3$
\end{itemize}
\item \emph{troubling} solution %$X \mapsto \tm{\lambda g.\lambda h.\pi_\li{3}}$
\begin{itemize}
\item $X\,G\,\pi_1 \mapsto \big(\lambda g.\lambda h.\pi_3\big)\,G\,\pi_1
\twoheadrightarrow_\beta \pi_3$
\end{itemize}
\item another \emph{troubling} solution %$X \mapsto \tm{\lambda g.\lambda h.\lambda xyz.h\,z\,z\,z}$
\begin{itemize}
\item $X\,G\,\pi_1 \mapsto \big(\lambda g.\lambda h.\lambda xyz.h\,z\,z\,z\big)\,G\,\pi_1
\twoheadrightarrow_\beta \lambda xyz.\pi_1\,z\,z\,z \twoheadrightarrow_\beta \pi_3$
\end{itemize}
\end{itemize}


\begin{itemize}
\item \emph{semantic} constraint (1) $X\,G\,\pi_1 =_\beta \pi_3$
\item \emph{syntactic} constraint (2) $X\,I\,u =_\beta u$
\item well-behaved solutions for (1) solve (2)
\begin{itemize}
\item $X\,I\,u \mapsto \big(\lambda g.\lambda h.g\,(g\,h)\big)\,I\,u
\twoheadrightarrow_\beta I\,(I\,u) \twoheadrightarrow_\beta u$
\end{itemize}
\item troubling solutions for (1) do \emph{not} solve (2)
\begin{itemize}
\item $X\,I\,u \mapsto \big(\lambda g.\lambda h.\pi_3\big)\,I\,u
\twoheadrightarrow_\beta \pi_3 \not=_\beta u$
\item $X\,I\,u \mapsto \big(\lambda g.\lambda h.\lambda xyz.h\,z\,z\,z\big)\,I\,u
\twoheadrightarrow_\beta \lambda xyz.u\,z\,z\,z \not=_\beta u$
\end{itemize}
\item combined matching instance $\big(\lambda x.\lambda y.y\,(x\,G\,\pi_1)\,(\lambda u.x\,I\,u)\big)\,X = \lambda y.y\,\pi_3\,(\lambda u.u)$
\end{itemize}
\end{example}



\begin{remark}~
\begin{itemize}
\setlength\itemsep{0.1em}
\item syntactic constraint restricts the number of $\lambda$-abstractions in solutions
\item solutions are \emph{not} $\eta$-long wrt. $(\sigma_\calU \to \sigma_\calU) \to \sigma_\calU \to \sigma_\calU$
\item solutions are inhabitants of intersection type $A$
%\item syntactic constraints don't work well with $\eta$-equality
\end{itemize}
\end{remark}

\begin{itemize}
\item example $\calR := \{00 \Rightarrow 21, 02 \Rightarrow 11\}$
\item atoms $\{\bot, \top, \bullet, \dag, 0, 1, 2 \}$
%\item universe $\calU := \{\tm{\pi}_\bot, \tm{\pi}_\top, \tm{\pi_\xt{\bullet}}, \tm{\pi_\li{0}}, \tm{\pi_\li{1}}, \tm{\pi_\li{0}}, \tm{\pi_\li{1}}, \tm{\pi_\xt{\dag}}, \tm{\pi_\li{0}}, \tm{\pi_\li{1}}, \tm{\pi_\li{2}} \}$
\item specification\\
{\small$A_\calR := A_{00 \Rightarrow 21} \to A_{02 \Rightarrow 11} \to A_{0} \to 1 \to A_\smile \to (\bullet^\top \to \dag) \wedge (0^\top \to 1) \wedge (1^\top \to 0)$}
\begin{itemize}
\item rule application\\
$A_{ab \Rightarrow cd} :=
\big(0^\top \to c \to a \big) \wedge \big(1^\top \to d \to b \big) \wedge \bigwedge\limits_{e \in \{0, 1, 2\}} \big(\bullet^\top \to e \to e \big)$
\item initialization\\
$A_{0} := \big(\bullet^\top \to 0 \to 0 \big) \wedge \big(1^\top \to 0 \to 0 \big) \wedge \big(0^\top \to 0 \to 1 \big) \wedge \big(\bullet^\top \to 1 \to \dag \big)$
\item finalization $1$
\item expansion (find suitable word length)\\
$\arraycolsep=1.4pt\begin{array}{lcl}
A_\smile &:=& \big(\bullet^\top \to \big(\bullet^\top \to 0 \big) \to 0 \big) \wedge \big(1^\top \to \big(\bullet^\top \to 0 \big) \to 0 \big) \wedge\phantom{pain}\\
&&\big(0^\top \to \big(1^\top \to 0 \big) \to 1 \big) \wedge \big(\bullet^\top \to \big(\big(0^\top \to 1 \big) \wedge \big(\bullet^\top \to \dag \big)\big) \to \dag \big)
\end{array}$
\item starting configuration $(\bullet^\top \to \dag) \wedge (0^\top \to 1) \wedge (1^\top \to 0)$
\end{itemize}
%\item finite function $f_\calR \in \enc{A_\calR}$
%\begin{itemize}
%\item $\enc{a} = \{a\}$ for $a \in \calU$
%\item $\enc{A \to B} = \{f \mid \text{for all } a \in \enc{A} \text{ we have } f(a) \in \enc{B} \}$
%\item $\enc{A \wedge B} = \enc{A} \wedge \enc{B}$
%\end{itemize}
\end{itemize}

\begin{itemize}
%\item example $\calR  = \{\li{00} \Rightarrow \li{21}, \li{02} \Rightarrow \li{11}\}$
\item universe $\calU := \{\pi_\bot, \pi_\top, \pi_\bullet, \pi_\dag, \pi_0, \pi_1, \pi_2 \}$
\item each $M \in \calU$ of simple type $\sigma_{\calU} := \underbrace{\ga \to \cdots \to \ga}_{\scriptsize|\calU| \text{ times}} \to \ga$
\item specification\\
{\small$A_\calR := A_{00 \Rightarrow 21} \to A_{02 \Rightarrow 11} \to A_{0} \to 1 \to A_\smile \to (\bullet^\top \to \dag) \wedge (0^\top \to 1) \wedge (1^\top \to 0)$}
\end{itemize}
%\begin{definition}[Interpolation Equation]
%$\tm{x\,M_1 \ldots M_n} = \tm{M_{n+1}}$ where $\tm{M_1}, \ldots, \tm{M_{n+1}}$ are terms in normal form with no occurrence of $\tm{x}$.
%\end{definition}
\begin{enumerate}
\item semantic constraints (1), (2), (3)
\begin{itemize}
\item[(1)] $X\,G_{00 \Rightarrow 21}\,G_{02 \Rightarrow 11}\,G_{0}\,\pi_{1}\,G_\smile\,G^{\top}_{\bullet} =_\beta \pi_{\dag}$
\item[(2)] $X\,G_{00 \Rightarrow 21}\,G_{02 \Rightarrow 11}\,G_{0}\,\pi_{1}\,G_\smile\,G^{\top}_{0} =_\beta \pi_{1}$
\item[(3)] $X\,G_{00 \Rightarrow 21}\,G_{02 \Rightarrow 11}\,G_{0}\,\pi_{1}\,G_\smile\,G^{\top}_{1} =_\beta \pi_{0}$
\end{itemize}
\item syntactic constraints (4), (5)
\begin{itemize}
\item[(4)] $X\,H_{00 \Rightarrow 21}\,H_{02 \Rightarrow 11}\,H_{0}\,\pi_{1}\,H_\smile\,H^{\top}_{\bullet} =_\beta \pi_{\dag}$
\item[(5)] $X\,I\,I\,(\lambda h.\lambda x.x)\,u\,(\lambda h.\lambda g.g\,I)\,I =_\beta u$
\end{itemize}
\item combine constraints (1)--(5) to matching instance $\calM_\calR := (F\,X = N)$
\begin{itemize}
\item $F := \lambda x.\lambda y.y\hspace*{-0.3em} \begin{array}[t]{l}
(x\,G_{00 \Rightarrow 21}\,G_{02 \Rightarrow 11}\,G_{0}\,\pi_{1}\,G_\smile\,G^{\top}_{\bullet})\\
(x\,G_{00 \Rightarrow 21}\,G_{02 \Rightarrow 11}\,G_{0}\,\pi_{1}\,G_\smile\,G^{\top}_{0})\\
(x\,G_{00 \Rightarrow 21}\,G_{02 \Rightarrow 11}\,G_{0}\,\pi_{1}\,G_\smile\,G^{\top}_{1})\\
(x\,H_{00 \Rightarrow 21}\,H_{02 \Rightarrow 11}\,H_{0}\,\pi_{1}\,H_\smile\,H^{\top}_{\bullet})\\
(\lambda u.x\,I\,I\,(\lambda h.\lambda x.x)\,u\,(\lambda h.\lambda g.g\,I)\,I)
\end{array}$
%$\begin{array}{lcll}
%\tm{F} &=& \lambda x.\lambda y.y & \big(da
%\end{array}$
\item $N := \lambda y.y\,\pi_{\dag}\,\pi_{1}\,\pi_{0}\,\pi_{\dag}\,(\lambda u.u)$
\end{itemize}
\end{enumerate}

\begin{theorem}
Given a simple semi-Thue system $\calR$, the matching instance $\calM_\calR$ is solvable iff for some $n > 0$ we have $0^n \Rightarrow^*_\calR 1^n$.
\end{theorem}

\begin{proof}[Proof Outline]
\vspace*{-0.2em}
\begin{itemize}
\item[$\Rightarrow$:] Induction on solution term size, using case analysis.
\item[$\Leftarrow$:] Induction on the number of rewrite steps.\qedhere
\end{itemize}
\end{proof}

\begin{corollary}
Higher-order $\beta$-matching is undecidable.
\end{corollary}

\begin{remark}
\begin{description}
\item[($30$ LOC)] problem definition (\lstinline$HOMbeta$)
\item[($1000$ LOC)] simply typed $\lambda$-calculus infrastructure (\lstinline$autosubst$, confluence, normalization) 
\item[($3000$ LOC)] many-one reduction from $0^+ \Rightarrow^*_\calR 1^+$ to \lstinline$HOMbeta$
\item[($4$s)] compilation time (heavy use of \lstinline$ssreflect$, \lstinline$lia$, \lstinline$auto$)
\end{description}
\end{remark}

\begin{itemize}
%\item universe elements $\tm{\pi_i} : \ity{\ga \to \cdots \to \ga \to \ga}$ of order $2$
\item $\beta$-matching is undecidable at order $6$~[\cite{Loader03}]
\begin{itemize}
\item $\lambda$-definability is undecidable at order $4$~[\cite{loader2001undecidability}]
\item $(7+1)$-lifts increase order by $2$
\end{itemize}
\item present work: $\beta$-matching at order $6$
\begin{itemize}
\item $A_\calR$ of order $5$
\item universe elements $\pi_i$ increase order by $1$
\end{itemize}
\item $\eta$-long $\beta$-matching is decidable at order $4$~[\cite{Padovani00}]
\item $\eta$-long $\beta$-matching is equivalent to $\beta\eta$-matching~[\cite{Stovring06}]
%\begin{itemize}
%\item translates to $\beta\eta$-matching at order $4$
%\item does decidability translate to $\beta$-matching at order $4$?
%\end{itemize}
\end{itemize}

\bigskip

\large
\enquote{\textit{one might guess that undecidability starts at order $5$}}~[\cite{Loader03}]
\normalsize

\bigskip

\begin{itemize}
\item undecidability at order $5$ using Loader's approach?
\begin{itemize}
\item remove $1$-lifts? (so far, I failed)
\end{itemize}
\item undecidability at order $5$ using present approach?
\begin{itemize}
\item intersection type inhabitation is undecidable at order $4$
\item universe elements $\pi_i$ increase order by $1$
\item specification argument $\top$ facilitates syntactic constraints
\item remove $\top$? (so far, I failed)
\end{itemize}
%\item undecidability at order $5$ using different approach?
%\begin{itemize}
%\item Hilbert's 10th problem as starting point
%\end{itemize}
\end{itemize}

\begin{example}
\begin{itemize}
\item specification $A := \big((1 \to 2) \wedge (2 \to 3) \wedge (3 \to 1)\big) \to 1 \to 3$
\item universe elements $\pi_1, \pi_2, \pi_3$
\item permutation $G$ realizing $(1 \to 2) \wedge (2 \to 3) \wedge (3 \to 1)$
\item \emph{semantic} constraint (1) $X\,G\,\pi_1 =_{\beta\eta} \pi_3$
\item \emph{syntactic} constraint (2) $X\,I\,u =_{\beta\eta} u$
\item troubling solution $X \mapsto \lambda g.\lambda h.\lambda xyz.h\,(g\,\pi_1 x\,z\,y)\,y\,z$
\begin{itemize}
\item[(1)] $\arraycolsep=1.4pt\begin{array}[t]{lll}
X\,G\,\pi_1 &\mapsto& \big(\lambda g.\lambda h.\lambda xyz.h\,(g\,\pi_1 x\,z\,y)\,y\,z\big)\,G\,\pi_1\\
&\twoheadrightarrow_\beta& \lambda xyz.\pi_{1}\,(G\,\pi_1 x\,z\,y)\,y\,z\\
&\twoheadrightarrow_\beta& \lambda xyz.\pi_2\,x\,z\,y \twoheadrightarrow_\beta \lambda xyz.z = \pi_{3}
\end{array}$
\item[(2)] $\arraycolsep=1.4pt\begin{array}[t]{lll}
X\,I\,u &\mapsto& \big(\lambda g.\lambda h.\lambda xyz.h\,(g\,\pi_1 x\,z\,y)\,y\,z\big)\,I\,u\\
&\twoheadrightarrow_\beta& \lambda xyz.u\,(I\,\pi_1 x\,z\,y)\,y\,z\\
&\twoheadrightarrow_\beta& \lambda xyz.u\,(\pi_1 x\,z\,y)\,y\,z \twoheadrightarrow_\beta \lambda xyz.u\,x\,y\,z =_\eta u
\end{array}$
\end{itemize}
\end{itemize}
\end{example}

\section{Introduction}

We like to reduce the problem $\exists n.\0^{n+1} \twoheadrightarrow \1^{1+n}$ for simple semi-Thue systems to higher-order matching.
A simple semi-Thue system $\calR = \{R_1, \ldots R_k\}$ over an alphabet $\Sigma \supseteq \{\0, \1\}$ contains rules of shape $ab \mapsto cd$.

For the following intersection types we have that the problem $\exists n.\0^{n+1} \twoheadrightarrow \1^{1+n}$ corresponds to the type inhabitation problem $\vdash\,?: \sigma_\star \to \sigma_\0 \to \sigma_{R_1} \to \cdots \sigma_{R_k} \to \sigma_\1 \to \$$.

\begin{align*}
\sigma_\star & := \Big(\big((\top \to \bullet) \to *\big) \to * \Big) \cap
\Big(\big((\top \to \RIGHTcircle) \to *\big) \to \# \Big) \cap
\bigg(\Big(\big((\top \to \LEFTcircle) \to \#\big) \cap \big((\top \to \bullet) \to \$ \big)\Big) \to \$ \bigg)\\
\sigma_0 & := \big(\0 \to (\top \to \bullet) \to * \big) \cap \big(\0 \to (\top \to \RIGHTcircle) \to * \big) \cap \big(\0 \to (\top \to \LEFTcircle) \to \# \big) \cap \big(\1 \to (\top \to \bullet) \to \$ \big)\\
\sigma_{ab \to cd} & := \big(c \to (\top \to \LEFTcircle) \to a \big) \cap \big(d \to (\top \to \RIGHTcircle) \to b \big) \cap \bigcap_{e \in \Sigma} \big(e \to (\top \to \bullet) \to e \big)\\
\sigma_1 & := \1
\end{align*}

\begin{remark}
The functions corresponding to $\sigma_\star, \sigma_\0, \sigma_{R_1}, \ldots, \sigma_{R_k}, \sigma_\1$ are realizable in the domain $D = \{\bullet, *, \LEFTcircle, \RIGHTcircle, \#, \$, \top, \bot\} \,\dot{\cup}\, \Sigma$.
\end{remark}

Intuition:
\begin{itemize}
\item $\sigma_\star$ expands a tape $* \ldots * \#\$$ and saves adjacency information.
\item $\sigma_\0$ initializes the tape to $\0 \ldots \0 \1$.
\item $\sigma_{ab \to cd}$ performs rewriting operations.
\item $\sigma_\1$ accepts the tape $\1 \ldots \1$.
\end{itemize}

\newpage

Let $\Sigma = \{e_1, \ldots, e_m\}$, $D := \{\bullet, *, \LEFTcircle, \RIGHTcircle, \#, \$, \top, \bot\} \,\dot{\cup}\, \Sigma$ and $n := |D| = 8 + m$ in the remainder of the presentation.
Let us assign unique values $1 \ldots n$ to individual symbols in $D$, where $\bot$ is assigned the value $n$.
We freely write $x$ instead of its assigned value.
For example, if $x \in D$ is assigned the value $i$ we write $r_x$ for $r_i$, and we write $\pi_x$ for $\lambda r_1 \ldots r_n.r_i$.


Fix the following simple type $A := \underbrace{0 \to \cdots 0 \to}_{n \text{ times}} 0$.

\[
\caseelse{x}{i_1 \mapsto t_1 \mid \ldots \mid i_k \mapsto t_k}{t} := (x\,u_1 \ldots u_n) \text{ where } u_i = \begin{cases} t_j & \text{if } i = i_j\\t & \text{otherwise}\end{cases}
\]

We may omit the \textbf{else} branch if it is the most recently bound variable, that is

\[\Big(\lambda r_1 \ldots r_n.\case{x}{i_1 \mapsto t_1 \mid \ldots \mid i_k \mapsto t_k}\Big) := \Big(\lambda r_1 \ldots r_n.\caseelse{x}{i_1 \mapsto t_1 \mid \ldots \mid i_k \mapsto t_k}{r_n}\Big)\]

If we only have one case and the \textbf{else} branch is the most recently bound variable we write

\[\Big(\lambda r_1 \ldots r_n.\itb{x}{i}{s}\Big) := \Big(\lambda r_1 \ldots r_n.\caseelse{x}{i \mapsto s}{r_n}\Big)\]

\newpage

\section{Shape Control}

consider the specification $\rho_{R_1} \to \cdots \to \rho_{R_k} \to \rho_0 \to \rho_1 \to \rho_\star \to \rho_\bullet \to \$$ where

\begin{align*}
\rho_\bullet & := \top \to \bullet \\
\rho_\star & := \rho_\bullet \to \Big(\rho_\bullet \to \$ \Big) \to \$\\
\rho_0 & := \rho_\bullet \to \1 \to \$\\
\rho_{ab \to cd} & := \rho_\bullet \to \1 \to \1\\
\rho_1 & := \1
\end{align*}

Inhabitants are necessary of shape $\lambda x_{R_1} \ldots x_{R_k} \, x_0 \, x_1 \, x_\star \, x_\bullet.M$ where $M$ is TODO

\begin{align*}
\delta^\top_i := & \lambda x.\lambda s_1 \ldots s_n.\itb{x}{\top}{s_i}\\\\
F_\star := &\lambda h^{A \to A}.g^{(A \to A) \to A}.g\,(\lambda x^A.x)\\
F_\0 := &\lambda h^{A \to A}.\lambda x^A.x\\
F_{ab \to cd} := &\lambda h^{A \to A}.h\\\\
H_\star := &\lambda h^{A \to A}.\lambda g^{(A \to A) \to A}.\lambda r_1 \ldots r_n. \itb{g\,\delta^\top_\bullet}{\$}{r_\$}\\
H_\0 := &\lambda h^{A \to A}.\lambda x^A. \lambda r_1 \ldots r_n.\itb{x}{\1}{r_\$}\\
H_{ab \to cd} := &\lambda h^{A \to A}.\lambda x^A. \lambda r_1 \ldots r_n.\itb{h\,\pi_\top}{{\bullet}}{(\itb{x}{\1}{r_\1})}\\
H_\1 := &\pi_\1
\end{align*}

Consider the higher-order matching problem

\begin{align*}
&X\,F_{R_1} \ldots F_{R_k}\,F_\0\,z\,F_\star\,I = z\\
&X\,H_{R_1} \ldots H_{R_k}\,H_\0\,H_\1\,H_\star\,\delta^\top_\bullet = \pi_\$
\end{align*}

for this the following constraint must be solved

\[\Big(\lambda x.\lambda y.y\,(\lambda z.x\,F_{R_1} \ldots F_{R_k}\,F_\0\,z\,F_\star\,I) (x\,H_{R_1} \ldots H_{R_k}\,H_\0\,H_\1\,H_\star\,\delta^\top_\bullet)\Big)\,X = \lambda y.y\,(\lambda z.z)\,\pi_\$\]

Solutions to this constraint have a specific shape which can be used for semantics.
The last $\delta^\top_\bullet$ is used for the first positional abstraction.

\newpage

\section{Semantics}

We mimic intersection type inhabitation using the following higher-order matching problem

\begin{align*}
&X\,G_{R_1} \ldots G_{R_k}\,G_\0\,G_\1\,G_\star\,\delta_\bullet = \pi_\$\\
&X\,G_{R_1} \ldots G_{R_k}\,G_\0\,G_\1\,G_\star\,\delta_{\LEFTcircle} = \pi_\#\\
&X\,G_{R_1} \ldots G_{R_k}\,G_\0\,G_\1\,G_\star\,\delta_{\RIGHTcircle} = \pi_*
\end{align*}
where
\begin{align*}
G_\star := &\lambda h^{A \to A}.\lambda g^{(A \to A) \to A}.\lambda r_1 \ldots r_n.\case{h\,\pi_\top}{\\
&\hspace{1.8em} \bullet \mapsto \case{g\,\delta^\top_\bullet}{* \mapsto r_* \mid \$ \mapsto \itb{g\,\delta^\top_{\LEFTcircle}}{\#}{r_\$}}\\
&\hspace{1.8em} \RIGHTcircle \mapsto \case{g\,\delta^\top_\bullet}{* \mapsto r_*}\\
&\hspace{1.8em} \LEFTcircle \mapsto \case{g\,\delta^\top_{\RIGHTcircle}}{* \mapsto r_\#}}\\
%&\quad\mid \$ \mapsto \caseelse{g\,\delta^\top_{\LEFTcircle}}{\# \mapsto r_\$}{r_\bot}}{\\
%&\quad \caseelse{g\,\delta^\top_{\RIGHTcircle}}{* \mapsto r_\#}{r_\bot}}\\
%&\caseelse{g\,\delta^\top_\bullet}{\\
%&\hspace{1.8em} * \mapsto r_*\\
%&\quad\mid \$ \mapsto \caseelse{g\,\delta^\top_{\LEFTcircle}}{\# \mapsto r_\$}{r_\bot}}{\\
%&\quad \caseelse{g\,\delta^\top_{\RIGHTcircle}}{* \mapsto r_\#}{r_\bot}}\\
G_\0 := &\lambda h^{A \to A}.\lambda x^A. \lambda r_1 \ldots r_n.\case{h\,\pi_\top}{\\
&\hspace{1.8em}\bullet \mapsto \case{x}{\0 \mapsto r_* \mid \1 \mapsto r_\$}\\
&\quad\mid\RIGHTcircle \mapsto \case{x}{\0 \mapsto r_*}\\
&\quad\mid\LEFTcircle \mapsto \case{x}{\0 \mapsto r_\# }}\\
G_{ab \to cd} := &\lambda h^{A \to A}.\lambda x^A.\lambda r_1 \ldots r_n.\case{h\,\pi_\top}{\\
&\hspace{1.8em}\bullet \mapsto x\,r_1 \ldots r_n\\
&\quad\mid\RIGHTcircle \mapsto \case{x}{d \mapsto r_b}\\
&\quad\mid\LEFTcircle \mapsto \case{x}{c \mapsto r_a }}\\
G_\1 := &\pi_\1
\end{align*}
recalling
\begin{align*}
\sigma_\star := & \Big((\top \to \bullet) \to \big((\top \to \bullet) \to *\big) \to * \Big) \cap \\
&\Big((\top \to \RIGHTcircle) \to \big((\top \to \bullet) \to *\big) \to * \Big) \cap \\
& \Big((\top \to \LEFTcircle) \to \big((\top \to \RIGHTcircle) \to *\big) \to \# \Big) \cap \\
& \bigg((\top \to \bullet) \to \Big(\big((\top \to \LEFTcircle) \to \#\big) \cap \big((\top \to \bullet) \to \$ \big)\Big) \to \$ \bigg)\\
\sigma_0 := &\big((\top \to \bullet) \to \0 \to * \big) \cap \big((\top \to \RIGHTcircle) \to \0 \to * \big) \cap \big((\top \to \LEFTcircle) \to \0 \to \# \big) \cap \big((\top \to \bullet) \to \1 \to \$ \big)\\
\sigma_{ab \to cd} := &\big((\top \to \LEFTcircle) \to c \to a \big) \cap \big((\top \to \RIGHTcircle) \to d \to b \big) \cap \bigcap_{e \in \Sigma} \big((\top \to \bullet) \to e \to e \big)\\
\sigma_1 := &\1
\end{align*}
all constraints are combined into the following matching problem over the variable $X$
\begin{align*}
\Big(\lambda x. \lambda y.y\,&(\lambda z.x\,F_{R_1} \ldots F_{R_k}\,F_\0\,z\,F_\star\,I)\\
&(x\,H_{R_1} \ldots H_{R_k}\,H_\0\,H_\1\,F_\star\,\delta_\bullet)\\
&(x\,G_{R_1} \ldots G_{R_k}\,G_\0\,G_\1\,G_\star\,\delta_\bullet)\\
&(x\,G_{R_1} \ldots G_{R_k}\,G_\0\,G_\1\,G_\star\,\delta_{\LEFTcircle})\\
&(x\,G_{R_1} \ldots G_{R_k}\,G_\0\,G_\1\,G_\star\,\delta_{\RIGHTcircle})\Big)\,X = \lambda y.y\,I\,\pi_\$\,\pi_\$\,\pi_\#\,\pi_*\\
\end{align*}

Remarks
\begin{itemize}
\item For inhabitation instead of $\top \to \LEFTcircle$ just the type $\LEFTcircle$ suffices, whereas for matching it is required for structural constraints (equation 1).
\end{itemize}

\section{Undecidability of Rank 3 Intersection Type Inhabitation}

Given a simple semi-Thue system $\calR$
\begin{enumerate}
\item construct intersection type $A_\calR$
\begin{itemize}
\item $A_\calR$ is inhabited iff $0^+ \Rightarrow^*_\calR 1^+$
\end{itemize}
\end{enumerate}

\bigskip

\begin{remark}[Properties of $A_\calR$]
\begin{itemize}
\item based on~[\cite{Urzyczyn09}] and~[\cite{DudenhefnerR19}]
\item has rank 3
\item allows for simple tools (no $\lambda\bot$-terms or myopic order)
\item $\lambda$-definable when interpreted as finite function~[\cite{SalvatiMGB12}]
\end{itemize}
\end{remark}

\begin{theorem}
Given a simple semi-Thue system $\calR$, the intersection type $A_\calR$ is inhabited iff for some $n > 0$ we have $0^n \Rightarrow^*_\calR 1^n$.
\end{theorem}

\begin{proof}[Proof Outline~{[\cite{DudenhefnerR19}]}]
\vspace*{-0.2em}
\begin{itemize}
\item[$\Rightarrow$:] Induction on inhabitant size, using generation lemma.
\item[$\Leftarrow$:] Induction on the number of rewrite steps.\qedhere
\end{itemize}
\end{proof}

\begin{corollary}
Intersection type inhabitation is undecidable.
\end{corollary}

\newpage


\bibliographystyle{plain} 
\bibliography{bibliography}

\end{document}
