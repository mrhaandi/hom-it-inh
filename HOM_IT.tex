\documentclass[a4paper,11pt]{article}
\usepackage[T1]{fontenc}\usepackage[utf8]{inputenc}
%%%%%%%%%%%%%%%%%%%%%%%%% Packages %%%%%%%%%%%%%%%%%%%%%%%%%%%%%%%%%%%%%%%%%
\usepackage{amssymb}
\usepackage{amsmath}
\usepackage{amsthm}
\usepackage{a4wide,xspace}
\usepackage{enumitem}
\usepackage{amsmath}
\usepackage{csquotes} %\enquote{} macro
\usepackage{lmodern} %easier on the eyes modern font
\usepackage{hyperref} %linked references
\usepackage{lineno}\linenumbers
\usepackage{wasysym} %\brokenvert
\usepackage[activate={true,nocompatibility}]{microtype} %layouting
%%%%%%%%%%%%%%%%%%%%%%% Style changes %%%%%%%%%%%%%%%%%%%%%%%%%%%%%%%%%%%
\theoremstyle{definition}
\newtheorem{theorem}{Theorem}
\newtheorem{lemma}[theorem]{Lemma}
\newtheorem{fact}[theorem]{Fact}
\newtheorem{corollary}[theorem]{Corollary}
\newtheorem{example}[theorem]{Example}
\newtheorem{claim}[theorem]{Claim}
\newtheorem{proposition}[theorem]{Proposition}
\newtheorem{remark}[theorem]{Remark}
\newtheorem{conjecture}[theorem]{Conjecture}
\newtheorem{definition}[theorem]{Definition}
\newtheorem{problem}[theorem]{Problem}
%%%%%%%%%%%%%%%%%%%%%%%%%%%% Symbols %%%%%%%%%%%%%%%%%%%%%%%%%%%%%%%%%%%

%%%%%%%%%%%%%%%%%% specific %%%%%%%%%%%%%%%%%%%%%%%%%%%%%%%%%%%%%%%%%%%%%%%
\DeclareMathOperator{\id}{id}
\DeclareMathOperator{\length}{length}
\DeclareMathOperator{\depth}{depth}


%%%%%%%%%%%%%%%%%%%%%%%% stylized symbols %%%%<%%%%%%%%%%%%%%%%%%%%%%%%%%%%%%%%%%%%
\newcommand{\bbB}{\mathbb{B}}
\newcommand{\bbC}{\mathbb{C}}
\newcommand{\bbN}{\mathbb{N}}
\newcommand{\bbT}{\mathbb{T}}
\newcommand{\bbS}{\mathbb{S}}
\newcommand{\bbV}{\mathbb{V}}
\newcommand{\bbZ}{\mathbb{Z}}
\newcommand{\calC}{\mathcal{C}}
\newcommand{\calD}{\mathcal{D}}
\newcommand{\calM}{\mathcal{M}}
\newcommand{\calR}{\mathcal{R}}
\newcommand{\calT}{\mathcal{T}}

\newcommand{\0}{\mathbf{0}}
\newcommand{\1}{\mathbf{1}}

\newcommand{\case}[2]{\ensuremath{\textbf{case } #1 \textbf{ of } \langle #2 \rangle}}
\newcommand{\caseelse}[3]{\ensuremath{\textbf{case } #1 \textbf{ of } \langle #2 \rangle \textbf{ else } #3}}
\newcommand{\ite}[4]{\ensuremath{\textbf{if } #1 \textbf{ is } #2 \textbf{ then } #3 \textbf{ else } #4}}
\newcommand{\itb}[3]{\ensuremath{\textbf{if } #1 \textbf{ is } #2 \textbf{ then } #3}}
%%%%%%%%%%%%%%%%%%%%%%%%% Abbreviations etc. %%%%%%%%%%%%%%%%%%%%%%%%%%%%%%%%

%%%%%%%%%%%%%%%%%%%%%%%%%%%%%%%%%%%%%%%%%%%%%%%%%%%%%%%%%%%%%%%%%%%%%%%%%%%%%
\begin{document}
\title{Higher-order Matching -- a Loader and Urzyczyn Story}
\author{Andrej Dudenhefner\\ TU Dortmund University, Germany \\
\texttt{andrej.dudenhefner@cs.tu-dortmund.de}}
\date{\today}
\maketitle

\begin{abstract}
Thoughts on undecidability of higher-order matching wrt.~$(=_\beta)$ (not to be confused with~$(=_{\beta\eta})$) using intersection type inhabitation.
\end{abstract}



\section{Introduction}

We like to reduce the problem $\exists n.\0^{n+1} \twoheadrightarrow \1^{1+n}$ for simple semi-Thue systems to higher-order matching.
A simple semi-Thue system $\calR = \{R_1, \ldots R_k\}$ over an alphabet $\Sigma \supseteq \{\0, \1\}$ contains rules of shape $ab \mapsto cd$.

For the following intersection types we have that the problem $\exists n.\0^{n+1} \twoheadrightarrow \1^{1+n}$ corresponds to the type inhabitation problem $\vdash\,?: \sigma_\star \to \sigma_\0 \to \sigma_{R_1} \to \cdots \sigma_{R_k} \to \sigma_\1 \to \$$.

\begin{align*}
\sigma_\star & := \Big(\big((\top \to \bullet) \to *\big) \to * \Big) \cap
\Big(\big((\top \to \RIGHTcircle) \to *\big) \to \# \Big) \cap
\bigg(\Big(\big((\top \to \LEFTcircle) \to \#\big) \cap \big((\top \to \bullet) \to \$ \big)\Big) \to \$ \bigg)\\
\sigma_0 & := \big(\0 \to (\top \to \bullet) \to * \big) \cap \big(\0 \to (\top \to \RIGHTcircle) \to * \big) \cap \big(\0 \to (\top \to \LEFTcircle) \to \# \big) \cap \big(\1 \to (\top \to \bullet) \to \$ \big)\\
\sigma_{ab \to cd} & := \big(c \to (\top \to \LEFTcircle) \to a \big) \cap \big(d \to (\top \to \RIGHTcircle) \to b \big) \cap \bigcap_{e \in \Sigma} \big(e \to (\top \to \bullet) \to e \big)\\
\sigma_1 & := \1
\end{align*}

\begin{remark}
The functions corresponding to $\sigma_\star, \sigma_\0, \sigma_{R_1}, \ldots, \sigma_{R_k}, \sigma_\1$ are realizable in the domain $D = \{\bullet, *, \LEFTcircle, \RIGHTcircle, \#, \$, \top, \bot\} \,\dot{\cup}\, \Sigma$.
\end{remark}

Intuition:
\begin{itemize}
\item $\sigma_\star$ expands a tape $* \ldots * \#\$$ and saves adjacency information.
\item $\sigma_\0$ initializes the tape to $\0 \ldots \0 \1$.
\item $\sigma_{ab \to cd}$ performs rewriting operations.
\item $\sigma_\1$ accepts the tape $\1 \ldots \1$.
\end{itemize}

\newpage

Let $\Sigma = \{e_1, \ldots, e_m\}$, $D := \{\bullet, *, \LEFTcircle, \RIGHTcircle, \#, \$, \top, \bot\} \,\dot{\cup}\, \Sigma$ and $n := |D| = 8 + m$ in the remainder of the presentation.
Let us assign unique values $1 \ldots n$ to individual symbols in $D$, where $\bot$ is assigned the value $n$.
We freely write $x$ instead of its assigned value.
For example, if $x \in D$ is assigned the value $i$ we write $r_x$ for $r_i$, and we write $\pi_x$ for $\lambda r_1 \ldots r_n.r_i$.


Fix the following simple type $A := \underbrace{0 \to \cdots 0 \to}_{n \text{ times}} 0$.

\[
\caseelse{x}{i_1 \mapsto t_1 \mid \ldots \mid i_k \mapsto t_k}{t} := (x\,u_1 \ldots u_n) \text{ where } u_i = \begin{cases} t_j & \text{if } i = i_j\\t & \text{otherwise}\end{cases}
\]

If we only have one case and the \textbf{else} branch is the most recently bound variable we write

\[\Big(\lambda r_1 \ldots r_n.\itb{x}{i}{s}\Big) := \Big(\lambda r_1 \ldots r_n.\caseelse{x}{i \mapsto s}{r_n}\Big)\]

\newpage

We mimic intersection type inhabitation using the following higher-order matching problem

\begin{align*}
&X\,F_\star\,F_\0\,F_{R_1} \ldots F_{R_k} = \lambda z.z\\
&X\,G_\star\,G_\0\,G_{R_1} \ldots G_{R_k}\,G_\1\,r_1\ldots r_n = r_\$
\end{align*}
where TODO actually need $r_\bot$ or $s_\bot$ in alternative syntax (no easy default?)
\begin{align*}
F_\star := &\lambda g^{(A \to A) \to A}.g\,(\lambda x^A.x)\\
F_\0 := &\lambda x^A. \lambda h^{A \to A}.h\,x\\
F_{ab \to cd} := &\lambda x^A. \lambda h^{A \to A}.h\,x\\\\
\delta^\top_i := & \lambda x.\lambda s_1 \ldots s_n.\caseelse{x}{\top \mapsto s_i}{s_\bot}\\
G_\star := &\lambda g^{(A \to A) \to A}.\lambda r_1 \ldots r_n. \caseelse{g\,\delta^\top_\bullet}{\\
&\hspace{1.8em} * \mapsto r_*\\
&\quad\mid \$ \mapsto \caseelse{g\,\delta^\top_{\LEFTcircle}}{\# \mapsto r_\$}{r_\bot}}{\\
&\quad \caseelse{g\,\delta^\top_{\RIGHTcircle}}{* \mapsto r_\#}{r_\bot}}\\
G_\0 := &\lambda x^A. \lambda h^{A \to A}.\lambda r_1 \ldots r_n.\caseelse{h\,\pi_\top}{\\
&\hspace{1.8em}\bullet \mapsto \caseelse{x}{\0 \mapsto r_* \mid \1 \mapsto r_\$}{r_\bot}\\
&\quad\mid\RIGHTcircle \mapsto \caseelse{x}{\0 \mapsto r_*}{r_\bot}\\
&\quad\mid\LEFTcircle \mapsto \caseelse{x}{\0 \mapsto r_\# }{r_\bot}}{r_\bot}\\
G_{ab \to cd} := &\lambda x^A. \lambda h^{A \to A}.\lambda r_1 \ldots r_n.\caseelse{h\,\pi_\top}{\\
&\hspace{1.8em}\bullet \mapsto \caseelse{x}{e_1 \mapsto r_{e_1} \mid \ldots \mid e_m \mapsto r_{e_m}}{r_\bot}\\
&\quad\mid\RIGHTcircle \mapsto \caseelse{x}{d \mapsto r_b}{r_\bot}\\
&\quad\mid\LEFTcircle \mapsto \caseelse{x}{c \mapsto r_a }{r_\bot}}{r_\bot}\\
G_\1 := &\pi_\1
\end{align*}
recalling
\begin{align*}
\sigma_\star & := \Big(\big((\top \to \bullet) \to *\big) \to * \Big) \cap
\Big(\big((\top \to \RIGHTcircle) \to *\big) \to \# \Big) \cap
\bigg(\Big(\big((\top \to \LEFTcircle) \to \#\big) \cap \big((\top \to \bullet) \to \$ \big)\Big) \to \$ \bigg)\\
\sigma_0 & := \big(\0 \to (\top \to \bullet) \to * \big) \cap \big(\0 \to (\top \to \RIGHTcircle) \to * \big) \cap \big(\0 \to (\top \to \LEFTcircle) \to \# \big) \cap \big(\1 \to (\top \to \bullet) \to \$ \big)\\
\sigma_{ab \to cd} & := \big(c \to (\top \to \LEFTcircle) \to a \big) \cap \big(d \to (\top \to \RIGHTcircle) \to b \big) \cap \bigcap_{e \in \Sigma} \big(e \to (\top \to \bullet) \to e \big)\\
\sigma_1 & := \1
\end{align*}
the two constraints are combined into the following matching problem over the variable $X$
\[(\lambda x.\lambda y.y\,(x\,F_\star\,F_\0\,F_{R_1} \ldots F_{R_k}) (\lambda r_1  \ldots r_n.x\,G_\star\,G_\0\,G_{R_1} \ldots G_{R_k}\,G_\1\,r_1 \ldots r_n))\,X = \lambda y.y\,(\lambda z.z)\,\pi_\$\]


Remarks
\begin{itemize}
\item For inhabitation instead of $\top \to \LEFTcircle$ just the type $\LEFTcircle$ suffices, whereas for matching it is required for structural constraints (equation 1).
\end{itemize}

\newpage

\section{Easier Proof}

consider the specification $\rho_{R_1} \to \cdots \to \rho_{R_k} \to \rho_0 \to \rho_\star \to \rho_1 \to \$$ where

\begin{align*}
\rho_\star & := \Big((\top \to \bullet) \to \$ \Big) \to \$\\
\rho_0 & := (\top \to \bullet) \to \1 \to \$\\
\rho_{ab \to cd} & := (\top \to \bullet) \to \1 \to \1\\
\rho_1 & := \1
\end{align*}

Inhabitants are necessary of shape $\lambda x_{R_1} \ldots x_{R_k} \, x_0 \, x_\star \, x_1.M$ where $M$ is TODO

\begin{align*}
\delta^\top_i := & \lambda x.\lambda s_1 \ldots s_n.\itb{x}{\top}{s_i}\\\\
F_\star := &\lambda g^{(A \to A) \to A}.g\,(\lambda x^A.x)\\
F_\0 := &\lambda h^{A \to A}.h\\
F_{ab \to cd} := &\lambda h^{A \to A}.h\\\\
H_\star := &\lambda g^{(A \to A) \to A}.\lambda r_1 \ldots r_n. \itb{g\,\delta^\top_\bullet}{\$}{r_\$}\\
H_\0 := &\lambda h^{A \to A}.\lambda x^A. \lambda r_1 \ldots r_n.\itb{h\,\pi_\top}{{\bullet}}{(\itb{x}{\1}{r_\$})}\\
H_{ab \to cd} := &\lambda h^{A \to A}.\lambda x^A. \lambda r_1 \ldots r_n.\itb{h\,\pi_\top}{{\bullet}}{(\itb{x}{\1}{r_\1})}\\
H_\1 := &\pi_\1
\end{align*}

Consider the higher-order matching problem

\begin{align*}
&X\,F_{R_1} \ldots F_{R_k}\,F_\0\,F_\star = \lambda z.z\\
&X\,H_{R_1} \ldots H_{R_k}\,H_\0\,H_\star\,H_\1\,q \ldots q\, p\, q \ldots q = p  \text{ where } p \text{ is at position } \$
\end{align*}

for this the following constraint must be solved

\[\Big(\lambda x.\lambda y.y\,(x\,F_{R_1} \ldots F_{R_k}\,F_\0\,F_\star) (\lambda p\,q.x\,G_{R_1} \ldots G_{R_k}\,G_\0\,G_\star\,G_\1\,q \ldots q\,p\,q \ldots q)\Big)\,X = \lambda y.y\,(\lambda z.z)\,(\lambda p\,q.p)\]

We write $M\,\vec{i}$ for $M\,t_1 \ldots t_n$ where $t_i := p$ and $t_j := q$ for $j \neq i$.
The above constraints are

\begin{align*}
&X\,F_{R_1} \ldots F_{R_k}\,F_\0\,F_\star = \lambda z.z\\
&X\,H_{R_1} \ldots H_{R_k}\,H_\0\,H_\star\,H_\1\,\vec{\$} = p
\end{align*}

and

\[\Big(\lambda x.\lambda y.y\,(x\,F_{R_1} \ldots F_{R_k}\,F_\0\,F_\star) (\lambda p\,q.x\,G_{R_1} \ldots G_{R_k}\,G_\0\,G_\star\,G_\1\,\vec{\$})\Big)\,X = \lambda y.y\,(\lambda z.z)\,(\lambda p\,q.p)\]

\bibliographystyle{plain} 
\bibliography{bibliography}

\end{document}
