\documentclass[a4paper,11pt]{article}
\usepackage[T1]{fontenc}\usepackage[utf8]{inputenc}
%%%%%%%%%%%%%%%%%%%%%%%%% Packages %%%%%%%%%%%%%%%%%%%%%%%%%%%%%%%%%%%%%%%%%
\usepackage{amssymb}
\usepackage{amsmath}
\usepackage{amsthm}
\usepackage{a4wide,xspace}
\usepackage{enumitem}
\usepackage{amsmath}
\usepackage{csquotes} %\enquote{} macro
\usepackage{lmodern} %easier on the eyes modern font
\usepackage{hyperref} %linked references
\usepackage{lineno}\linenumbers
\usepackage{wasysym} %\brokenvert
\usepackage[activate={true,nocompatibility}]{microtype} %layouting
%%%%%%%%%%%%%%%%%%%%%%% Style changes %%%%%%%%%%%%%%%%%%%%%%%%%%%%%%%%%%%
\theoremstyle{definition}
\newtheorem{theorem}{Theorem}
\newtheorem{lemma}[theorem]{Lemma}
\newtheorem{fact}[theorem]{Fact}
\newtheorem{corollary}[theorem]{Corollary}
\newtheorem{example}[theorem]{Example}
\newtheorem{claim}[theorem]{Claim}
\newtheorem{proposition}[theorem]{Proposition}
\newtheorem{remark}[theorem]{Remark}
\newtheorem{conjecture}[theorem]{Conjecture}
\newtheorem{definition}[theorem]{Definition}
\newtheorem{problem}[theorem]{Problem}
%%%%%%%%%%%%%%%%%%%%%%%%%%%% Symbols %%%%%%%%%%%%%%%%%%%%%%%%%%%%%%%%%%%

%%%%%%%%%%%%%%%%%% specific %%%%%%%%%%%%%%%%%%%%%%%%%%%%%%%%%%%%%%%%%%%%%%%
\DeclareMathOperator{\id}{id}
\DeclareMathOperator{\length}{length}
\DeclareMathOperator{\depth}{depth}


%%%%%%%%%%%%%%%%%%%%%%%% stylized symbols %%%%<%%%%%%%%%%%%%%%%%%%%%%%%%%%%%%%%%%%%
\newcommand{\bbB}{\mathbb{B}}
\newcommand{\bbC}{\mathbb{C}}
\newcommand{\bbN}{\mathbb{N}}
\newcommand{\bbT}{\mathbb{T}}
\newcommand{\bbS}{\mathbb{S}}
\newcommand{\bbV}{\mathbb{V}}
\newcommand{\bbZ}{\mathbb{Z}}
\newcommand{\calC}{\mathcal{C}}
\newcommand{\calD}{\mathcal{D}}
\newcommand{\calM}{\mathcal{M}}
\newcommand{\calR}{\mathcal{R}}
\newcommand{\calT}{\mathcal{T}}

\newcommand{\0}{\mathbf{0}}
\newcommand{\1}{\mathbf{1}}
%%%%%%%%%%%%%%%%%%%%%%%%% Abbreviations etc. %%%%%%%%%%%%%%%%%%%%%%%%%%%%%%%%

%%%%%%%%%%%%%%%%%%%%%%%%%%%%%%%%%%%%%%%%%%%%%%%%%%%%%%%%%%%%%%%%%%%%%%%%%%%%%
\begin{document}
\title{Title}
\author{Andrej Dudenhefner\\ TU Dortmund University, Germany \\
\texttt{andrej.dudenhefner@cs.tu-dortmund.de}}
\date{\today}
\maketitle

\begin{abstract}
Thoughts on undecidability of higher-order matching wrt.~$(=_\beta)$ (not to be confused with~$(=_{\beta\eta})$) using intersection type inhabitation.
\end{abstract}



\section{Introduction}

We like to reduce the problem $\exists n.\0^{n+1} \twoheadrightarrow \1^{1+n}$ for simple semi-Thue systems to higher-order matching.
A simple semi-Thue system $\calR = \{R_1, \ldots R_k\}$ over an alphabet $\Sigma \supseteq \{\0, \1\}$ contains rules of shape $ab \mapsto cd$.

For the following intersection types we have that the problem $\exists n.\0^{n+1} \twoheadrightarrow \1^{1+n}$ corresponds to the type inhabitation problem $\vdash\,?: \sigma_\star \to \sigma_\0 \to \sigma_{R_1} \to \cdots \sigma_{R_k} \to \sigma_\1 \to \$$.

\begin{align*}
\sigma_\star & := \Big(\big((\1 \to \bullet) \to *\big) \to * \Big) \cap
\Big(\big((\1 \to \RIGHTcircle) \to *\big) \to \# \Big) \cap
\bigg(\Big(\big((\1 \to \LEFTcircle) \to \#\big) \cap \big((\1 \to \bullet) \to \$ \big)\Big) \to \$ \bigg)\\
\sigma_0 & := \big(\0 \to (\1 \to \bullet) \to * \big) \cap \big(\0 \to (\1 \to \RIGHTcircle) \to * \big) \cap \big(\0 \to (\1 \to \LEFTcircle) \to \# \big) \cap \big(\1 \to (\1 \to \bullet) \to \$ \big)\\
\sigma_{ab \to cd} & := \big(c \to (\1 \to \LEFTcircle) \to a \big) \cap \big(d \to (\1 \to \RIGHTcircle) \to b \big) \cap \bigcap_{e \in \Sigma} \big(e \to (\1 \to \bullet) \to e \big)\\
\sigma_1 & := \1
\end{align*}

\begin{remark}
The functions corresponding to $\sigma_\star, \sigma_\0, \sigma_{R_1}, \ldots, \sigma_{R_k}, \sigma_\1$ are realizable in the domain $D = \{\bullet, *, \LEFTcircle, \RIGHTcircle, \#, \$\} \,\dot{\cup}\, \Sigma$.
\end{remark}

Intuition:
\begin{itemize}
\item $\sigma_\star$ expands a tape $* \ldots * \#\$$ and saves adjacency information.
\item $\sigma_\0$ initializes the tape to $\0 \ldots \0 \1$.
\item $\sigma_{ab \to cd}$ performs rewriting operations.
\item $\sigma_\1$ accepts the tape $\1 \ldots \1$.
\end{itemize}

\newpage

Let $D := \{\bullet, *, \LEFTcircle, \RIGHTcircle, \#, \$\} \,\dot{\cup}\, \Sigma$ and $n := |D|$ in the remainder of the presentation.
Let us also assign unique values $1 \ldots n$ to individual symbols in $D$.
For $x \in D$ with the assigned value $i$ to $x$ we write $r_x$ for $r_i$ and we write $\pi_x$ for $\lambda r_1 \ldots r_n.r_i$.

Let $A := \underbrace{0 \to \cdots 0 \to}_{n \text{ times}} 0$.

\begin{align*}
&\textbf{match } x \textbf{ with}\\
&|\, y_1 \Rightarrow t_1\\
&\ldots\\
&|\, y_k \Rightarrow t_k\\
&|\, \_ \Rightarrow t\\
&\textbf{end}\\
&:=
&\lambda r_1 \ldots r_n. (x\,r_1 \ldots r_n)[r_i := if i = ?]
\end{align*}

\[\textbf{if } x \textbf{ is } y \textbf{ then } u \textbf{ else } v := \lambda r_1 \ldots r_n. \]

We mimic inhabitation using the higher-order matching problem


\begin{align*}
&X\,F_\star\,F_\0\,F_{R_1} \ldots F_{R_k} = \lambda z.z\\
&X\,G_\star\,G_\0\,G_{R_1} \ldots G_{R_k}\,G_\1 = \pi_\$\\
&\text{where}\\
&F_\star := \lambda g^{(A \to A) \to A}.g\,(\lambda x^A.x)\\
&F_\0 := \lambda x^A. \lambda h^{A \to A}.h\,x\\
&F_{ab \to cd} := \lambda x^A. \lambda h^{A \to A}.h\,x\\
&G_\star := ?\\
&G_\0 := \lambda x^A. \lambda h^{A \to A}.\lambda r_1 \ldots r_n.h\,\1\,t_1\ldots t_n\\
&\text{where}\\
&t_\bullet := x\,u_1\ldots u_n\\
&G_{ab \to cd} := ?\\
&G_\1 := \pi_\1
\end{align*}


\bibliographystyle{plain} 
\bibliography{bibliography}

\end{document}
